\documentclass[11pt]{article}

    \usepackage[breakable]{tcolorbox}
    \usepackage{parskip} % Stop auto-indenting (to mimic markdown behaviour)
    
    \usepackage{iftex}
    \ifPDFTeX
    	\usepackage[T1]{fontenc}
    	\usepackage{mathpazo}
    \else
    	\usepackage{fontspec}
    \fi

    % Basic figure setup, for now with no caption control since it's done
    % automatically by Pandoc (which extracts ![](path) syntax from Markdown).
    \usepackage{graphicx}
    % Maintain compatibility with old templates. Remove in nbconvert 6.0
    \let\Oldincludegraphics\includegraphics
    % Ensure that by default, figures have no caption (until we provide a
    % proper Figure object with a Caption API and a way to capture that
    % in the conversion process - todo).
    \usepackage{caption}
    \DeclareCaptionFormat{nocaption}{}
    \captionsetup{format=nocaption,aboveskip=0pt,belowskip=0pt}

    \usepackage{float}
    \floatplacement{figure}{H} % forces figures to be placed at the correct location
    \usepackage{xcolor} % Allow colors to be defined
    \usepackage{enumerate} % Needed for markdown enumerations to work
    \usepackage{geometry} % Used to adjust the document margins
    \usepackage{amsmath} % Equations
    \usepackage{amssymb} % Equations
    \usepackage{textcomp} % defines textquotesingle
    % Hack from http://tex.stackexchange.com/a/47451/13684:
    \AtBeginDocument{%
        \def\PYZsq{\textquotesingle}% Upright quotes in Pygmentized code
    }
    \usepackage{upquote} % Upright quotes for verbatim code
    \usepackage{eurosym} % defines \euro
    \usepackage[mathletters]{ucs} % Extended unicode (utf-8) support
    \usepackage{fancyvrb} % verbatim replacement that allows latex
    \usepackage{grffile} % extends the file name processing of package graphics 
                         % to support a larger range
    \makeatletter % fix for old versions of grffile with XeLaTeX
    \@ifpackagelater{grffile}{2019/11/01}
    {
      % Do nothing on new versions
    }
    {
      \def\Gread@@xetex#1{%
        \IfFileExists{"\Gin@base".bb}%
        {\Gread@eps{\Gin@base.bb}}%
        {\Gread@@xetex@aux#1}%
      }
    }
    \makeatother
    \usepackage[Export]{adjustbox} % Used to constrain images to a maximum size
    \adjustboxset{max size={0.9\linewidth}{0.9\paperheight}}

    % The hyperref package gives us a pdf with properly built
    % internal navigation ('pdf bookmarks' for the table of contents,
    % internal cross-reference links, web links for URLs, etc.)
    \usepackage{hyperref}
    % The default LaTeX title has an obnoxious amount of whitespace. By default,
    % titling removes some of it. It also provides customization options.
    \usepackage{titling}
    \usepackage{longtable} % longtable support required by pandoc >1.10
    \usepackage{booktabs}  % table support for pandoc > 1.12.2
    \usepackage[inline]{enumitem} % IRkernel/repr support (it uses the enumerate* environment)
    \usepackage[normalem]{ulem} % ulem is needed to support strikethroughs (\sout)
                                % normalem makes italics be italics, not underlines
    \usepackage{mathrsfs}
    

    
    % Colors for the hyperref package
    \definecolor{urlcolor}{rgb}{0,.145,.698}
    \definecolor{linkcolor}{rgb}{.71,0.21,0.01}
    \definecolor{citecolor}{rgb}{.12,.54,.11}

    % ANSI colors
    \definecolor{ansi-black}{HTML}{3E424D}
    \definecolor{ansi-black-intense}{HTML}{282C36}
    \definecolor{ansi-red}{HTML}{E75C58}
    \definecolor{ansi-red-intense}{HTML}{B22B31}
    \definecolor{ansi-green}{HTML}{00A250}
    \definecolor{ansi-green-intense}{HTML}{007427}
    \definecolor{ansi-yellow}{HTML}{DDB62B}
    \definecolor{ansi-yellow-intense}{HTML}{B27D12}
    \definecolor{ansi-blue}{HTML}{208FFB}
    \definecolor{ansi-blue-intense}{HTML}{0065CA}
    \definecolor{ansi-magenta}{HTML}{D160C4}
    \definecolor{ansi-magenta-intense}{HTML}{A03196}
    \definecolor{ansi-cyan}{HTML}{60C6C8}
    \definecolor{ansi-cyan-intense}{HTML}{258F8F}
    \definecolor{ansi-white}{HTML}{C5C1B4}
    \definecolor{ansi-white-intense}{HTML}{A1A6B2}
    \definecolor{ansi-default-inverse-fg}{HTML}{FFFFFF}
    \definecolor{ansi-default-inverse-bg}{HTML}{000000}

    % common color for the border for error outputs.
    \definecolor{outerrorbackground}{HTML}{FFDFDF}

    % commands and environments needed by pandoc snippets
    % extracted from the output of `pandoc -s`
    \providecommand{\tightlist}{%
      \setlength{\itemsep}{0pt}\setlength{\parskip}{0pt}}
    \DefineVerbatimEnvironment{Highlighting}{Verbatim}{commandchars=\\\{\}}
    % Add ',fontsize=\small' for more characters per line
    \newenvironment{Shaded}{}{}
    \newcommand{\KeywordTok}[1]{\textcolor[rgb]{0.00,0.44,0.13}{\textbf{{#1}}}}
    \newcommand{\DataTypeTok}[1]{\textcolor[rgb]{0.56,0.13,0.00}{{#1}}}
    \newcommand{\DecValTok}[1]{\textcolor[rgb]{0.25,0.63,0.44}{{#1}}}
    \newcommand{\BaseNTok}[1]{\textcolor[rgb]{0.25,0.63,0.44}{{#1}}}
    \newcommand{\FloatTok}[1]{\textcolor[rgb]{0.25,0.63,0.44}{{#1}}}
    \newcommand{\CharTok}[1]{\textcolor[rgb]{0.25,0.44,0.63}{{#1}}}
    \newcommand{\StringTok}[1]{\textcolor[rgb]{0.25,0.44,0.63}{{#1}}}
    \newcommand{\CommentTok}[1]{\textcolor[rgb]{0.38,0.63,0.69}{\textit{{#1}}}}
    \newcommand{\OtherTok}[1]{\textcolor[rgb]{0.00,0.44,0.13}{{#1}}}
    \newcommand{\AlertTok}[1]{\textcolor[rgb]{1.00,0.00,0.00}{\textbf{{#1}}}}
    \newcommand{\FunctionTok}[1]{\textcolor[rgb]{0.02,0.16,0.49}{{#1}}}
    \newcommand{\RegionMarkerTok}[1]{{#1}}
    \newcommand{\ErrorTok}[1]{\textcolor[rgb]{1.00,0.00,0.00}{\textbf{{#1}}}}
    \newcommand{\NormalTok}[1]{{#1}}
    
    % Additional commands for more recent versions of Pandoc
    \newcommand{\ConstantTok}[1]{\textcolor[rgb]{0.53,0.00,0.00}{{#1}}}
    \newcommand{\SpecialCharTok}[1]{\textcolor[rgb]{0.25,0.44,0.63}{{#1}}}
    \newcommand{\VerbatimStringTok}[1]{\textcolor[rgb]{0.25,0.44,0.63}{{#1}}}
    \newcommand{\SpecialStringTok}[1]{\textcolor[rgb]{0.73,0.40,0.53}{{#1}}}
    \newcommand{\ImportTok}[1]{{#1}}
    \newcommand{\DocumentationTok}[1]{\textcolor[rgb]{0.73,0.13,0.13}{\textit{{#1}}}}
    \newcommand{\AnnotationTok}[1]{\textcolor[rgb]{0.38,0.63,0.69}{\textbf{\textit{{#1}}}}}
    \newcommand{\CommentVarTok}[1]{\textcolor[rgb]{0.38,0.63,0.69}{\textbf{\textit{{#1}}}}}
    \newcommand{\VariableTok}[1]{\textcolor[rgb]{0.10,0.09,0.49}{{#1}}}
    \newcommand{\ControlFlowTok}[1]{\textcolor[rgb]{0.00,0.44,0.13}{\textbf{{#1}}}}
    \newcommand{\OperatorTok}[1]{\textcolor[rgb]{0.40,0.40,0.40}{{#1}}}
    \newcommand{\BuiltInTok}[1]{{#1}}
    \newcommand{\ExtensionTok}[1]{{#1}}
    \newcommand{\PreprocessorTok}[1]{\textcolor[rgb]{0.74,0.48,0.00}{{#1}}}
    \newcommand{\AttributeTok}[1]{\textcolor[rgb]{0.49,0.56,0.16}{{#1}}}
    \newcommand{\InformationTok}[1]{\textcolor[rgb]{0.38,0.63,0.69}{\textbf{\textit{{#1}}}}}
    \newcommand{\WarningTok}[1]{\textcolor[rgb]{0.38,0.63,0.69}{\textbf{\textit{{#1}}}}}
    
    
    % Define a nice break command that doesn't care if a line doesn't already
    % exist.
    \def\br{\hspace*{\fill} \\* }
    % Math Jax compatibility definitions
    \def\gt{>}
    \def\lt{<}
    \let\Oldtex\TeX
    \let\Oldlatex\LaTeX
    \renewcommand{\TeX}{\textrm{\Oldtex}}
    \renewcommand{\LaTeX}{\textrm{\Oldlatex}}
    % Document parameters
    % Document title
    \title{Actividad\_1\_EL\_Amazon\_Robots}
    
    
    
    
    
% Pygments definitions
\makeatletter
\def\PY@reset{\let\PY@it=\relax \let\PY@bf=\relax%
    \let\PY@ul=\relax \let\PY@tc=\relax%
    \let\PY@bc=\relax \let\PY@ff=\relax}
\def\PY@tok#1{\csname PY@tok@#1\endcsname}
\def\PY@toks#1+{\ifx\relax#1\empty\else%
    \PY@tok{#1}\expandafter\PY@toks\fi}
\def\PY@do#1{\PY@bc{\PY@tc{\PY@ul{%
    \PY@it{\PY@bf{\PY@ff{#1}}}}}}}
\def\PY#1#2{\PY@reset\PY@toks#1+\relax+\PY@do{#2}}

\expandafter\def\csname PY@tok@w\endcsname{\def\PY@tc##1{\textcolor[rgb]{0.73,0.73,0.73}{##1}}}
\expandafter\def\csname PY@tok@c\endcsname{\let\PY@it=\textit\def\PY@tc##1{\textcolor[rgb]{0.25,0.50,0.50}{##1}}}
\expandafter\def\csname PY@tok@cp\endcsname{\def\PY@tc##1{\textcolor[rgb]{0.74,0.48,0.00}{##1}}}
\expandafter\def\csname PY@tok@k\endcsname{\let\PY@bf=\textbf\def\PY@tc##1{\textcolor[rgb]{0.00,0.50,0.00}{##1}}}
\expandafter\def\csname PY@tok@kp\endcsname{\def\PY@tc##1{\textcolor[rgb]{0.00,0.50,0.00}{##1}}}
\expandafter\def\csname PY@tok@kt\endcsname{\def\PY@tc##1{\textcolor[rgb]{0.69,0.00,0.25}{##1}}}
\expandafter\def\csname PY@tok@o\endcsname{\def\PY@tc##1{\textcolor[rgb]{0.40,0.40,0.40}{##1}}}
\expandafter\def\csname PY@tok@ow\endcsname{\let\PY@bf=\textbf\def\PY@tc##1{\textcolor[rgb]{0.67,0.13,1.00}{##1}}}
\expandafter\def\csname PY@tok@nb\endcsname{\def\PY@tc##1{\textcolor[rgb]{0.00,0.50,0.00}{##1}}}
\expandafter\def\csname PY@tok@nf\endcsname{\def\PY@tc##1{\textcolor[rgb]{0.00,0.00,1.00}{##1}}}
\expandafter\def\csname PY@tok@nc\endcsname{\let\PY@bf=\textbf\def\PY@tc##1{\textcolor[rgb]{0.00,0.00,1.00}{##1}}}
\expandafter\def\csname PY@tok@nn\endcsname{\let\PY@bf=\textbf\def\PY@tc##1{\textcolor[rgb]{0.00,0.00,1.00}{##1}}}
\expandafter\def\csname PY@tok@ne\endcsname{\let\PY@bf=\textbf\def\PY@tc##1{\textcolor[rgb]{0.82,0.25,0.23}{##1}}}
\expandafter\def\csname PY@tok@nv\endcsname{\def\PY@tc##1{\textcolor[rgb]{0.10,0.09,0.49}{##1}}}
\expandafter\def\csname PY@tok@no\endcsname{\def\PY@tc##1{\textcolor[rgb]{0.53,0.00,0.00}{##1}}}
\expandafter\def\csname PY@tok@nl\endcsname{\def\PY@tc##1{\textcolor[rgb]{0.63,0.63,0.00}{##1}}}
\expandafter\def\csname PY@tok@ni\endcsname{\let\PY@bf=\textbf\def\PY@tc##1{\textcolor[rgb]{0.60,0.60,0.60}{##1}}}
\expandafter\def\csname PY@tok@na\endcsname{\def\PY@tc##1{\textcolor[rgb]{0.49,0.56,0.16}{##1}}}
\expandafter\def\csname PY@tok@nt\endcsname{\let\PY@bf=\textbf\def\PY@tc##1{\textcolor[rgb]{0.00,0.50,0.00}{##1}}}
\expandafter\def\csname PY@tok@nd\endcsname{\def\PY@tc##1{\textcolor[rgb]{0.67,0.13,1.00}{##1}}}
\expandafter\def\csname PY@tok@s\endcsname{\def\PY@tc##1{\textcolor[rgb]{0.73,0.13,0.13}{##1}}}
\expandafter\def\csname PY@tok@sd\endcsname{\let\PY@it=\textit\def\PY@tc##1{\textcolor[rgb]{0.73,0.13,0.13}{##1}}}
\expandafter\def\csname PY@tok@si\endcsname{\let\PY@bf=\textbf\def\PY@tc##1{\textcolor[rgb]{0.73,0.40,0.53}{##1}}}
\expandafter\def\csname PY@tok@se\endcsname{\let\PY@bf=\textbf\def\PY@tc##1{\textcolor[rgb]{0.73,0.40,0.13}{##1}}}
\expandafter\def\csname PY@tok@sr\endcsname{\def\PY@tc##1{\textcolor[rgb]{0.73,0.40,0.53}{##1}}}
\expandafter\def\csname PY@tok@ss\endcsname{\def\PY@tc##1{\textcolor[rgb]{0.10,0.09,0.49}{##1}}}
\expandafter\def\csname PY@tok@sx\endcsname{\def\PY@tc##1{\textcolor[rgb]{0.00,0.50,0.00}{##1}}}
\expandafter\def\csname PY@tok@m\endcsname{\def\PY@tc##1{\textcolor[rgb]{0.40,0.40,0.40}{##1}}}
\expandafter\def\csname PY@tok@gh\endcsname{\let\PY@bf=\textbf\def\PY@tc##1{\textcolor[rgb]{0.00,0.00,0.50}{##1}}}
\expandafter\def\csname PY@tok@gu\endcsname{\let\PY@bf=\textbf\def\PY@tc##1{\textcolor[rgb]{0.50,0.00,0.50}{##1}}}
\expandafter\def\csname PY@tok@gd\endcsname{\def\PY@tc##1{\textcolor[rgb]{0.63,0.00,0.00}{##1}}}
\expandafter\def\csname PY@tok@gi\endcsname{\def\PY@tc##1{\textcolor[rgb]{0.00,0.63,0.00}{##1}}}
\expandafter\def\csname PY@tok@gr\endcsname{\def\PY@tc##1{\textcolor[rgb]{1.00,0.00,0.00}{##1}}}
\expandafter\def\csname PY@tok@ge\endcsname{\let\PY@it=\textit}
\expandafter\def\csname PY@tok@gs\endcsname{\let\PY@bf=\textbf}
\expandafter\def\csname PY@tok@gp\endcsname{\let\PY@bf=\textbf\def\PY@tc##1{\textcolor[rgb]{0.00,0.00,0.50}{##1}}}
\expandafter\def\csname PY@tok@go\endcsname{\def\PY@tc##1{\textcolor[rgb]{0.53,0.53,0.53}{##1}}}
\expandafter\def\csname PY@tok@gt\endcsname{\def\PY@tc##1{\textcolor[rgb]{0.00,0.27,0.87}{##1}}}
\expandafter\def\csname PY@tok@err\endcsname{\def\PY@bc##1{\setlength{\fboxsep}{0pt}\fcolorbox[rgb]{1.00,0.00,0.00}{1,1,1}{\strut ##1}}}
\expandafter\def\csname PY@tok@kc\endcsname{\let\PY@bf=\textbf\def\PY@tc##1{\textcolor[rgb]{0.00,0.50,0.00}{##1}}}
\expandafter\def\csname PY@tok@kd\endcsname{\let\PY@bf=\textbf\def\PY@tc##1{\textcolor[rgb]{0.00,0.50,0.00}{##1}}}
\expandafter\def\csname PY@tok@kn\endcsname{\let\PY@bf=\textbf\def\PY@tc##1{\textcolor[rgb]{0.00,0.50,0.00}{##1}}}
\expandafter\def\csname PY@tok@kr\endcsname{\let\PY@bf=\textbf\def\PY@tc##1{\textcolor[rgb]{0.00,0.50,0.00}{##1}}}
\expandafter\def\csname PY@tok@bp\endcsname{\def\PY@tc##1{\textcolor[rgb]{0.00,0.50,0.00}{##1}}}
\expandafter\def\csname PY@tok@fm\endcsname{\def\PY@tc##1{\textcolor[rgb]{0.00,0.00,1.00}{##1}}}
\expandafter\def\csname PY@tok@vc\endcsname{\def\PY@tc##1{\textcolor[rgb]{0.10,0.09,0.49}{##1}}}
\expandafter\def\csname PY@tok@vg\endcsname{\def\PY@tc##1{\textcolor[rgb]{0.10,0.09,0.49}{##1}}}
\expandafter\def\csname PY@tok@vi\endcsname{\def\PY@tc##1{\textcolor[rgb]{0.10,0.09,0.49}{##1}}}
\expandafter\def\csname PY@tok@vm\endcsname{\def\PY@tc##1{\textcolor[rgb]{0.10,0.09,0.49}{##1}}}
\expandafter\def\csname PY@tok@sa\endcsname{\def\PY@tc##1{\textcolor[rgb]{0.73,0.13,0.13}{##1}}}
\expandafter\def\csname PY@tok@sb\endcsname{\def\PY@tc##1{\textcolor[rgb]{0.73,0.13,0.13}{##1}}}
\expandafter\def\csname PY@tok@sc\endcsname{\def\PY@tc##1{\textcolor[rgb]{0.73,0.13,0.13}{##1}}}
\expandafter\def\csname PY@tok@dl\endcsname{\def\PY@tc##1{\textcolor[rgb]{0.73,0.13,0.13}{##1}}}
\expandafter\def\csname PY@tok@s2\endcsname{\def\PY@tc##1{\textcolor[rgb]{0.73,0.13,0.13}{##1}}}
\expandafter\def\csname PY@tok@sh\endcsname{\def\PY@tc##1{\textcolor[rgb]{0.73,0.13,0.13}{##1}}}
\expandafter\def\csname PY@tok@s1\endcsname{\def\PY@tc##1{\textcolor[rgb]{0.73,0.13,0.13}{##1}}}
\expandafter\def\csname PY@tok@mb\endcsname{\def\PY@tc##1{\textcolor[rgb]{0.40,0.40,0.40}{##1}}}
\expandafter\def\csname PY@tok@mf\endcsname{\def\PY@tc##1{\textcolor[rgb]{0.40,0.40,0.40}{##1}}}
\expandafter\def\csname PY@tok@mh\endcsname{\def\PY@tc##1{\textcolor[rgb]{0.40,0.40,0.40}{##1}}}
\expandafter\def\csname PY@tok@mi\endcsname{\def\PY@tc##1{\textcolor[rgb]{0.40,0.40,0.40}{##1}}}
\expandafter\def\csname PY@tok@il\endcsname{\def\PY@tc##1{\textcolor[rgb]{0.40,0.40,0.40}{##1}}}
\expandafter\def\csname PY@tok@mo\endcsname{\def\PY@tc##1{\textcolor[rgb]{0.40,0.40,0.40}{##1}}}
\expandafter\def\csname PY@tok@ch\endcsname{\let\PY@it=\textit\def\PY@tc##1{\textcolor[rgb]{0.25,0.50,0.50}{##1}}}
\expandafter\def\csname PY@tok@cm\endcsname{\let\PY@it=\textit\def\PY@tc##1{\textcolor[rgb]{0.25,0.50,0.50}{##1}}}
\expandafter\def\csname PY@tok@cpf\endcsname{\let\PY@it=\textit\def\PY@tc##1{\textcolor[rgb]{0.25,0.50,0.50}{##1}}}
\expandafter\def\csname PY@tok@c1\endcsname{\let\PY@it=\textit\def\PY@tc##1{\textcolor[rgb]{0.25,0.50,0.50}{##1}}}
\expandafter\def\csname PY@tok@cs\endcsname{\let\PY@it=\textit\def\PY@tc##1{\textcolor[rgb]{0.25,0.50,0.50}{##1}}}

\def\PYZbs{\char`\\}
\def\PYZus{\char`\_}
\def\PYZob{\char`\{}
\def\PYZcb{\char`\}}
\def\PYZca{\char`\^}
\def\PYZam{\char`\&}
\def\PYZlt{\char`\<}
\def\PYZgt{\char`\>}
\def\PYZsh{\char`\#}
\def\PYZpc{\char`\%}
\def\PYZdl{\char`\$}
\def\PYZhy{\char`\-}
\def\PYZsq{\char`\'}
\def\PYZdq{\char`\"}
\def\PYZti{\char`\~}
% for compatibility with earlier versions
\def\PYZat{@}
\def\PYZlb{[}
\def\PYZrb{]}
\makeatother


    % For linebreaks inside Verbatim environment from package fancyvrb. 
    \makeatletter
        \newbox\Wrappedcontinuationbox 
        \newbox\Wrappedvisiblespacebox 
        \newcommand*\Wrappedvisiblespace {\textcolor{red}{\textvisiblespace}} 
        \newcommand*\Wrappedcontinuationsymbol {\textcolor{red}{\llap{\tiny$\m@th\hookrightarrow$}}} 
        \newcommand*\Wrappedcontinuationindent {3ex } 
        \newcommand*\Wrappedafterbreak {\kern\Wrappedcontinuationindent\copy\Wrappedcontinuationbox} 
        % Take advantage of the already applied Pygments mark-up to insert 
        % potential linebreaks for TeX processing. 
        %        {, <, #, %, $, ' and ": go to next line. 
        %        _, }, ^, &, >, - and ~: stay at end of broken line. 
        % Use of \textquotesingle for straight quote. 
        \newcommand*\Wrappedbreaksatspecials {% 
            \def\PYGZus{\discretionary{\char`\_}{\Wrappedafterbreak}{\char`\_}}% 
            \def\PYGZob{\discretionary{}{\Wrappedafterbreak\char`\{}{\char`\{}}% 
            \def\PYGZcb{\discretionary{\char`\}}{\Wrappedafterbreak}{\char`\}}}% 
            \def\PYGZca{\discretionary{\char`\^}{\Wrappedafterbreak}{\char`\^}}% 
            \def\PYGZam{\discretionary{\char`\&}{\Wrappedafterbreak}{\char`\&}}% 
            \def\PYGZlt{\discretionary{}{\Wrappedafterbreak\char`\<}{\char`\<}}% 
            \def\PYGZgt{\discretionary{\char`\>}{\Wrappedafterbreak}{\char`\>}}% 
            \def\PYGZsh{\discretionary{}{\Wrappedafterbreak\char`\#}{\char`\#}}% 
            \def\PYGZpc{\discretionary{}{\Wrappedafterbreak\char`\%}{\char`\%}}% 
            \def\PYGZdl{\discretionary{}{\Wrappedafterbreak\char`\$}{\char`\$}}% 
            \def\PYGZhy{\discretionary{\char`\-}{\Wrappedafterbreak}{\char`\-}}% 
            \def\PYGZsq{\discretionary{}{\Wrappedafterbreak\textquotesingle}{\textquotesingle}}% 
            \def\PYGZdq{\discretionary{}{\Wrappedafterbreak\char`\"}{\char`\"}}% 
            \def\PYGZti{\discretionary{\char`\~}{\Wrappedafterbreak}{\char`\~}}% 
        } 
        % Some characters . , ; ? ! / are not pygmentized. 
        % This macro makes them "active" and they will insert potential linebreaks 
        \newcommand*\Wrappedbreaksatpunct {% 
            \lccode`\~`\.\lowercase{\def~}{\discretionary{\hbox{\char`\.}}{\Wrappedafterbreak}{\hbox{\char`\.}}}% 
            \lccode`\~`\,\lowercase{\def~}{\discretionary{\hbox{\char`\,}}{\Wrappedafterbreak}{\hbox{\char`\,}}}% 
            \lccode`\~`\;\lowercase{\def~}{\discretionary{\hbox{\char`\;}}{\Wrappedafterbreak}{\hbox{\char`\;}}}% 
            \lccode`\~`\:\lowercase{\def~}{\discretionary{\hbox{\char`\:}}{\Wrappedafterbreak}{\hbox{\char`\:}}}% 
            \lccode`\~`\?\lowercase{\def~}{\discretionary{\hbox{\char`\?}}{\Wrappedafterbreak}{\hbox{\char`\?}}}% 
            \lccode`\~`\!\lowercase{\def~}{\discretionary{\hbox{\char`\!}}{\Wrappedafterbreak}{\hbox{\char`\!}}}% 
            \lccode`\~`\/\lowercase{\def~}{\discretionary{\hbox{\char`\/}}{\Wrappedafterbreak}{\hbox{\char`\/}}}% 
            \catcode`\.\active
            \catcode`\,\active 
            \catcode`\;\active
            \catcode`\:\active
            \catcode`\?\active
            \catcode`\!\active
            \catcode`\/\active 
            \lccode`\~`\~ 	
        }
    \makeatother

    \let\OriginalVerbatim=\Verbatim
    \makeatletter
    \renewcommand{\Verbatim}[1][1]{%
        %\parskip\z@skip
        \sbox\Wrappedcontinuationbox {\Wrappedcontinuationsymbol}%
        \sbox\Wrappedvisiblespacebox {\FV@SetupFont\Wrappedvisiblespace}%
        \def\FancyVerbFormatLine ##1{\hsize\linewidth
            \vtop{\raggedright\hyphenpenalty\z@\exhyphenpenalty\z@
                \doublehyphendemerits\z@\finalhyphendemerits\z@
                \strut ##1\strut}%
        }%
        % If the linebreak is at a space, the latter will be displayed as visible
        % space at end of first line, and a continuation symbol starts next line.
        % Stretch/shrink are however usually zero for typewriter font.
        \def\FV@Space {%
            \nobreak\hskip\z@ plus\fontdimen3\font minus\fontdimen4\font
            \discretionary{\copy\Wrappedvisiblespacebox}{\Wrappedafterbreak}
            {\kern\fontdimen2\font}%
        }%
        
        % Allow breaks at special characters using \PYG... macros.
        \Wrappedbreaksatspecials
        % Breaks at punctuation characters . , ; ? ! and / need catcode=\active 	
        \OriginalVerbatim[#1,codes*=\Wrappedbreaksatpunct]%
    }
    \makeatother

    % Exact colors from NB
    \definecolor{incolor}{HTML}{303F9F}
    \definecolor{outcolor}{HTML}{D84315}
    \definecolor{cellborder}{HTML}{CFCFCF}
    \definecolor{cellbackground}{HTML}{F7F7F7}
    
    % prompt
    \makeatletter
    \newcommand{\boxspacing}{\kern\kvtcb@left@rule\kern\kvtcb@boxsep}
    \makeatother
    \newcommand{\prompt}[4]{
        {\ttfamily\llap{{\color{#2}[#3]:\hspace{3pt}#4}}\vspace{-\baselineskip}}
    }
    

    
    % Prevent overflowing lines due to hard-to-break entities
    \sloppy 
    % Setup hyperref package
    \hypersetup{
      breaklinks=true,  % so long urls are correctly broken across lines
      colorlinks=true,
      urlcolor=urlcolor,
      linkcolor=linkcolor,
      citecolor=citecolor,
      }
    % Slightly bigger margins than the latex defaults
    
    \geometry{verbose,tmargin=1in,bmargin=1in,lmargin=1in,rmargin=1in}
    
    

\begin{document}
    
    \maketitle
    
    

    
    \hypertarget{actividad-resoluciuxf3n-de-problema-mediante-buxfasqueda-heuruxedstica}{%
\section{Actividad : Resolución de problema mediante búsqueda
heurística}\label{actividad-resoluciuxf3n-de-problema-mediante-buxfasqueda-heuruxedstica}}

\hypertarget{objetivos-de-la-actividad}{%
\subsection{Objetivos de la actividad}\label{objetivos-de-la-actividad}}

Con esta actividad vas a conseguir implementar la estrategia de búsqueda
heurística A* para la resolución de un problema real.

\hypertarget{descripciuxf3n-de-la-actividad}{%
\subsection{Descripción de la
actividad}\label{descripciuxf3n-de-la-actividad}}

\begin{figure}
\centering
\includegraphics{https://drive.google.com/uc?export=view\&id=1nku190wKb1Wn8V-d_X8RVoREuKjavF0I}
\caption{texto alternativo}
\end{figure}

La empresa Amazon desea utilizar un robot para ordenar el inventario de
su almacén.

Amazon cuenta con 3 inventarios (mesa con suministros para vender)
localizados en unas posiciones específicas del almacén. El robot se debe
encargar de mover los 3 inventarios a una posición objetivo.

El robot puede moverse horizontal y verticalmente, y cargar o descargar
un inventario.

Un ejemplo del robot, moviendo el inventario, se puede observar en el
siguiente vídeo:

    \begin{tcolorbox}[breakable, size=fbox, boxrule=1pt, pad at break*=1mm,colback=cellbackground, colframe=cellborder]
\prompt{In}{incolor}{ }{\boxspacing}
\begin{Verbatim}[commandchars=\\\{\}]
\PY{c+c1}{\PYZsh{}from IPython.display import YouTubeVideo}
\PY{c+c1}{\PYZsh{}YouTubeVideo(\PYZsq{}UtBa9yVZBJM\PYZsq{})}
\end{Verbatim}
\end{tcolorbox}

    En esta actividad has de utilizar la estrategia de búsqueda heurística
A* con el fin de generar un plan que permita al robot de Amazon mover el
inventario de un estado inicial a un estado objetivo.

    \hypertarget{estado-inicial}{%
\subsection{Estado inicial}\label{estado-inicial}}

El estado inicial del problema lo vamos a representar en una mátriz 4x4
de carácteres de la siguiente manera:

\begin{figure}
\centering
\includegraphics{https://drive.google.com/uc?export=view\&id=1Mzml6909YsiwqCB77-xeLYRoVt93uLKg}
\caption{texto alternativo}
\end{figure}

Donde,

\begin{itemize}
\tightlist
\item
  R: representa el robot. Inicialmente está ubicado en la posición
  {[}2,2{]}
\item
  \#: representa una pared.
\item
  M1, M2, e M3: representan los tres inventarios que el robot debe
  mover. Y se encuentran ubicadas en las posiciones {[}0,0{]}, {[}2,0{]}
  y {[}0,3{]} respectivamente.
\end{itemize}

\hypertarget{estado-objetivo}{%
\subsection{Estado Objetivo}\label{estado-objetivo}}

El robot debe mover los 3 inventarios, M1, M2 y M3, a la siguientes
posiciones:

\begin{figure}
\centering
\includegraphics{https://drive.google.com/uc?export=view\&id=1rhaD4HXNAHjp9v2algygF38-OVNePBSe}
\caption{texto alternativo}
\end{figure}

    \hypertarget{tareas-a-realizar}{%
\subsection{Tareas a realizar}\label{tareas-a-realizar}}

Implementar el algoritmo A* considerando lo siguiente:

\begin{enumerate}
\def\labelenumi{\arabic{enumi}.}
\tightlist
\item
  como función heurística, la Distancia en Manhattan.
\item
  el coste real (c) de cada acción del robot es 1.
\item
  el código deberá ejecutarse e indicar la secuencia de acciones a
  realizar para alcanzar el estado objetivo utilizando una notación
  sencilla. Por ejemplo: «mover R fila1 columna2» o «mover R fila0
  columna2» o «cargar R M1 fila0 columna2».
\end{enumerate}

\hypertarget{documentos-a-entregar}{%
\subsection{Documentos a entregar}\label{documentos-a-entregar}}

\begin{itemize}
\item
  Memoria en word o jupyter notebook explicando en detalle el desarrollo
  de la actividad. Se recomienda un limite máximo de 10 páginas sin
  contar el código fuente. La memoria como mínimo debe contenter:
\item
  Portada
\item
  Desarrollo de la actividad: análisis, pantallazos de ejecución,
  pruebas realizadas, plan de acción.
\item
  Dificultades encontradas
\item
  Referencias bibliografícas con Normas APA
\item
  Código fuente desarrollado correctamente documentado.
\end{itemize}

\hypertarget{consideraciones-finales}{%
\subsection{Consideraciones finales}\label{consideraciones-finales}}

Puede seleccionar el lenguaje de programación que usted desee.

Se deberán crear tantas clases o estructuras de datos como sean
necesarias para representar el espacio de estados y los nodos de
exploración del árbol.

El programa desarrollado debe ser un trabajo original del estudiante.
Cualquier evidencia de o trabajos iguales será calificada con una nota
de cero (0).

    \begin{tcolorbox}[breakable, size=fbox, boxrule=1pt, pad at break*=1mm,colback=cellbackground, colframe=cellborder]
\prompt{In}{incolor}{ }{\boxspacing}
\begin{Verbatim}[commandchars=\\\{\}]
\PY{o}{\PYZpc{}}\PY{k}{matplotlib} inline

\PY{k+kn}{import} \PY{n+nn}{numpy} \PY{k}{as} \PY{n+nn}{np}
\PY{k+kn}{import} \PY{n+nn}{matplotlib}
\PY{n}{matplotlib}\PY{o}{.}\PY{n}{use}\PY{p}{(}\PY{l+s+s2}{\PYZdq{}}\PY{l+s+s2}{Agg}\PY{l+s+s2}{\PYZdq{}}\PY{p}{)}
\PY{k+kn}{import} \PY{n+nn}{matplotlib}\PY{n+nn}{.}\PY{n+nn}{pyplot} \PY{k}{as} \PY{n+nn}{plt}
\PY{k+kn}{import} \PY{n+nn}{matplotlib}\PY{n+nn}{.}\PY{n+nn}{animation} \PY{k}{as} \PY{n+nn}{animation}
\end{Verbatim}
\end{tcolorbox}

    \hypertarget{portada}{%
\section{Portada}\label{portada}}

\begin{center}\rule{0.5\linewidth}{0.5pt}\end{center}

\#\#Planificación y Razonamiento Automático

\#\#\#Jorge Augusto Balsells Orellana

\#\#\#Erick Wilfredo Díaz Saborio

\begin{center}\rule{0.5\linewidth}{0.5pt}\end{center}

\#\#\#\#31 de Enero de 2021

    \hypertarget{desarrollo-de-actividad}{%
\section{Desarrollo de Actividad}\label{desarrollo-de-actividad}}

    

    \hypertarget{definiciuxf3n-del-tablero}{%
\subsection{Definición del Tablero}\label{definiciuxf3n-del-tablero}}

Vamos a definir una matiz de valores numericos con numpy. Ya que los
valores de la matriz son numericos se definen 2 diccionarios de python
para poder mapear estos valores.

    \begin{tcolorbox}[breakable, size=fbox, boxrule=1pt, pad at break*=1mm,colback=cellbackground, colframe=cellborder]
\prompt{In}{incolor}{ }{\boxspacing}
\begin{Verbatim}[commandchars=\\\{\}]
\PY{n}{num\PYZus{}object\PYZus{}mapping} \PY{o}{=} \PY{p}{\PYZob{}}
    \PY{l+m+mi}{2}\PY{p}{:} \PY{l+s+s1}{\PYZsq{}}\PY{l+s+s1}{robot}\PY{l+s+s1}{\PYZsq{}}\PY{p}{,}
    \PY{l+m+mi}{3}\PY{p}{:} \PY{l+s+s1}{\PYZsq{}}\PY{l+s+s1}{M1}\PY{l+s+s1}{\PYZsq{}}\PY{p}{,}
    \PY{l+m+mi}{4}\PY{p}{:} \PY{l+s+s1}{\PYZsq{}}\PY{l+s+s1}{M2}\PY{l+s+s1}{\PYZsq{}}\PY{p}{,}
    \PY{l+m+mi}{5}\PY{p}{:} \PY{l+s+s1}{\PYZsq{}}\PY{l+s+s1}{M3}\PY{l+s+s1}{\PYZsq{}}\PY{p}{,}
    \PY{l+m+mi}{1}\PY{p}{:} \PY{l+s+s1}{\PYZsq{}}\PY{l+s+s1}{pared}\PY{l+s+s1}{\PYZsq{}}\PY{p}{,}
    \PY{l+m+mi}{0}\PY{p}{:} \PY{l+s+s1}{\PYZsq{}}\PY{l+s+s1}{espacio}\PY{l+s+s1}{\PYZsq{}}
\PY{p}{\PYZcb{}}

\PY{n}{object\PYZus{}num\PYZus{}mapping} \PY{o}{=} \PY{n+nb}{dict}\PY{p}{(}\PY{n+nb}{map}\PY{p}{(}\PY{n+nb}{reversed}\PY{p}{,} \PY{n}{num\PYZus{}object\PYZus{}mapping}\PY{o}{.}\PY{n}{items}\PY{p}{(}\PY{p}{)}\PY{p}{)}\PY{p}{)}
\end{Verbatim}
\end{tcolorbox}

    \begin{tcolorbox}[breakable, size=fbox, boxrule=1pt, pad at break*=1mm,colback=cellbackground, colframe=cellborder]
\prompt{In}{incolor}{ }{\boxspacing}
\begin{Verbatim}[commandchars=\\\{\}]
\PY{n}{object\PYZus{}num\PYZus{}mapping}
\end{Verbatim}
\end{tcolorbox}

            \begin{tcolorbox}[breakable, size=fbox, boxrule=.5pt, pad at break*=1mm, opacityfill=0]
\prompt{Out}{outcolor}{ }{\boxspacing}
\begin{Verbatim}[commandchars=\\\{\}]
\{'M1': 3, 'M2': 4, 'M3': 5, 'espacio': 0, 'pared': 1, 'robot': 2\}
\end{Verbatim}
\end{tcolorbox}
        
    \hypertarget{euclidean-distance}{%
\subsection{Euclidean Distance}\label{euclidean-distance}}

\(d=\sqrt{(x_1 - x_2)^2 + (y_1-y_2)^2}\)

    \begin{tcolorbox}[breakable, size=fbox, boxrule=1pt, pad at break*=1mm,colback=cellbackground, colframe=cellborder]
\prompt{In}{incolor}{ }{\boxspacing}
\begin{Verbatim}[commandchars=\\\{\}]
\PY{k}{def} \PY{n+nf}{euclidean\PYZus{}distance}\PY{p}{(}\PY{n}{pos1\PYZus{}x}\PY{p}{,} \PY{n}{pos1\PYZus{}y}\PY{p}{,} \PY{n}{pos2\PYZus{}x}\PY{p}{,} \PY{n}{pos2\PYZus{}y}\PY{p}{)}\PY{p}{:}
     \PY{k}{return} \PY{n}{np}\PY{o}{.}\PY{n}{sqrt}\PY{p}{(}\PY{n+nb}{pow}\PY{p}{(}\PY{n+nb}{abs}\PY{p}{(}\PY{n}{pos2\PYZus{}x}\PY{o}{\PYZhy{}}\PY{n}{pos1\PYZus{}x}\PY{p}{)}\PY{p}{,}\PY{l+m+mi}{2}\PY{p}{)}\PY{o}{+}\PY{n+nb}{pow}\PY{p}{(}\PY{n+nb}{abs}\PY{p}{(}\PY{n}{pos2\PYZus{}y}\PY{o}{\PYZhy{}}\PY{n}{pos1\PYZus{}y}\PY{p}{)}\PY{p}{,}\PY{l+m+mi}{2}\PY{p}{)}\PY{p}{)}
\end{Verbatim}
\end{tcolorbox}

    \hypertarget{heuristica-f}{%
\subsection{Heuristica f()}\label{heuristica-f}}

Función de estimación de la distancia entre la posición actual y el nodo
final. Esta función recibe como parámetros una matriz, la posición
actual y la destino.

Calcula los valores de \(g\), \(f\) y \(h\) de las celdas vecinas a la
posición actual, únicamente si es valido moverse a ellas, es decir si la
celda vecina esta vacía o es la posición final.

La función retorna una matriz, cada elemento de la matriz es un vector
con los valores de \(g\), \(f\), \(h\), \([x, y]\) si podemos movernos a
la celda en caso contrario retorna un arreglo vacío.

    \begin{tcolorbox}[breakable, size=fbox, boxrule=1pt, pad at break*=1mm,colback=cellbackground, colframe=cellborder]
\prompt{In}{incolor}{ }{\boxspacing}
\begin{Verbatim}[commandchars=\\\{\}]
\PY{k}{def} \PY{n+nf}{fgh\PYZus{}values}\PY{p}{(}\PY{n}{matrix}\PY{p}{,} \PY{n}{target}\PY{p}{,} \PY{n}{position}\PY{p}{)}\PY{p}{:}

    \PY{c+c1}{\PYZsh{}Guarda la posicion en coordenadas (x,y) del lugar donde esta ubicado}
    \PY{c+c1}{\PYZsh{}en position y de la carga en target}
    \PY{p}{[}\PY{n}{pos\PYZus{}y}\PY{p}{,} \PY{n}{pos\PYZus{}x}\PY{p}{]} \PY{o}{=} \PY{n}{position}
    \PY{p}{[}\PY{n}{tar\PYZus{}y}\PY{p}{,} \PY{n}{tar\PYZus{}x}\PY{p}{]} \PY{o}{=} \PY{n}{target}

    \PY{c+c1}{\PYZsh{}se determinan las dimensiones de filas y columnas del territorio o matriz}
    \PY{n}{mat\PYZus{}x} \PY{o}{=} \PY{n}{np}\PY{o}{.}\PY{n}{size}\PY{p}{(}\PY{n}{initial\PYZus{}state}\PY{p}{,}\PY{l+m+mi}{0}\PY{p}{)}
    \PY{n}{mat\PYZus{}y} \PY{o}{=} \PY{n}{np}\PY{o}{.}\PY{n}{size}\PY{p}{(}\PY{n}{initial\PYZus{}state}\PY{p}{,}\PY{l+m+mi}{1}\PY{p}{)}

    \PY{c+c1}{\PYZsh{}se crea una matriz de 3x3, que contendra los valores de f,g y h de todo el}
    \PY{c+c1}{\PYZsh{}perimetro del valor ingresado.}
    \PY{n}{f\PYZus{}mat} \PY{o}{=} \PY{p}{[}\PY{p}{[}\PY{l+m+mi}{0}\PY{p}{,}\PY{l+m+mi}{0}\PY{p}{,}\PY{l+m+mi}{0}\PY{p}{]}\PY{p}{,}\PY{p}{[}\PY{l+m+mi}{0}\PY{p}{,}\PY{l+m+mi}{0}\PY{p}{,}\PY{l+m+mi}{0}\PY{p}{]}\PY{p}{,}\PY{p}{[}\PY{l+m+mi}{0}\PY{p}{,}\PY{l+m+mi}{0}\PY{p}{,}\PY{l+m+mi}{0}\PY{p}{]}\PY{p}{]}

    \PY{c+c1}{\PYZsh{}se operan los 9 valores a ingresar en la matriz f\PYZus{}mat}
    \PY{k}{for} \PY{n}{x} \PY{o+ow}{in} \PY{n+nb}{range}\PY{p}{(}\PY{n}{pos\PYZus{}x}\PY{o}{\PYZhy{}}\PY{l+m+mi}{1}\PY{p}{,}\PY{n}{pos\PYZus{}x}\PY{o}{+}\PY{l+m+mi}{2}\PY{p}{)}\PY{p}{:}
        \PY{k}{for} \PY{n}{y} \PY{o+ow}{in} \PY{n+nb}{range}\PY{p}{(}\PY{n}{pos\PYZus{}y}\PY{o}{\PYZhy{}}\PY{l+m+mi}{1}\PY{p}{,}\PY{n}{pos\PYZus{}y}\PY{o}{+}\PY{l+m+mi}{2}\PY{p}{)}\PY{p}{:}

            \PY{c+c1}{\PYZsh{}se calculan los valores solamente si el punto esta dentro de las}
            \PY{c+c1}{\PYZsh{}dimensiones de la matriz, o es un espacio, o son las coordenadas}
            \PY{c+c1}{\PYZsh{}de la carga a la que se dirige el robot}
            \PY{k}{if}\PY{p}{(}\PY{n}{x}\PY{o}{\PYZgt{}}\PY{o}{=}\PY{l+m+mi}{0} \PY{o+ow}{and} \PY{n}{x}\PY{o}{\PYZlt{}}\PY{o}{=}\PY{n}{mat\PYZus{}x}\PY{o}{\PYZhy{}}\PY{l+m+mi}{1} \PY{o+ow}{and} \PY{n}{y}\PY{o}{\PYZgt{}}\PY{o}{=}\PY{l+m+mi}{0} \PY{o+ow}{and} \PY{n}{y}\PY{o}{\PYZlt{}}\PY{o}{=}\PY{n}{mat\PYZus{}y}\PY{o}{\PYZhy{}}\PY{l+m+mi}{1} 
               \PY{o+ow}{and} \PY{p}{(}\PY{p}{(}\PY{n}{matrix}\PY{p}{[}\PY{n}{y}\PY{p}{,}\PY{n}{x}\PY{p}{]}\PY{o}{==}\PY{n}{object\PYZus{}num\PYZus{}mapping}\PY{p}{[}\PY{l+s+s1}{\PYZsq{}}\PY{l+s+s1}{espacio}\PY{l+s+s1}{\PYZsq{}}\PY{p}{]}\PY{p}{)} \PY{o+ow}{or} \PY{p}{(}\PY{p}{(}\PY{n}{x}\PY{p}{,}\PY{n}{y}\PY{p}{)} \PY{o}{==} \PY{p}{(}\PY{n}{tar\PYZus{}x}\PY{p}{,} \PY{n}{tar\PYZus{}y}\PY{p}{)}\PY{p}{)}\PY{p}{)} \PY{p}{)}\PY{p}{:}

                \PY{c+c1}{\PYZsh{}Calculo de variables}
                \PY{n}{g} \PY{o}{=} \PY{n}{euclidean\PYZus{}distance}\PY{p}{(}\PY{n}{pos\PYZus{}x}\PY{p}{,} \PY{n}{pos\PYZus{}y}\PY{p}{,} \PY{n}{x}\PY{p}{,} \PY{n}{y}\PY{p}{)}
                \PY{n}{h} \PY{o}{=} \PY{n+nb}{abs}\PY{p}{(}\PY{n}{tar\PYZus{}y}\PY{o}{\PYZhy{}}\PY{n}{y}\PY{p}{)} \PY{o}{+} \PY{n+nb}{abs}\PY{p}{(}\PY{n}{tar\PYZus{}x}\PY{o}{\PYZhy{}}\PY{n}{x}\PY{p}{)}
                \PY{n}{f} \PY{o}{=} \PY{n}{g} \PY{o}{+} \PY{n}{h}

                \PY{c+c1}{\PYZsh{}Se crea un solo vector que se guarda en cada posicion de la matriz f\PYZus{}mat}
                \PY{n}{f\PYZus{}mat}\PY{p}{[}\PY{n}{x}\PY{o}{\PYZhy{}}\PY{n}{pos\PYZus{}x}\PY{o}{+}\PY{l+m+mi}{1}\PY{p}{]}\PY{p}{[}\PY{n}{y}\PY{o}{\PYZhy{}}\PY{n}{pos\PYZus{}y}\PY{o}{+}\PY{l+m+mi}{1}\PY{p}{]} \PY{o}{=} \PY{p}{[}\PY{n+nb}{round}\PY{p}{(}\PY{n}{f}\PY{p}{,}\PY{l+m+mi}{1}\PY{p}{)}\PY{p}{,} \PY{n+nb}{round}\PY{p}{(}\PY{n}{g}\PY{p}{,}\PY{l+m+mi}{1}\PY{p}{)}\PY{p}{,} \PY{n+nb}{round}\PY{p}{(}\PY{n}{h}\PY{p}{,}\PY{l+m+mi}{1}\PY{p}{)}\PY{p}{,} \PY{p}{[}\PY{n}{x}\PY{p}{,}\PY{n}{y}\PY{p}{]}\PY{p}{]}
            \PY{k}{else}\PY{p}{:}

                \PY{c+c1}{\PYZsh{}Si no se cumple la sentencia, guarda un vector vacio en esa posicion}
                \PY{n}{f\PYZus{}mat}\PY{p}{[}\PY{n}{x}\PY{o}{\PYZhy{}}\PY{n}{pos\PYZus{}x}\PY{o}{+}\PY{l+m+mi}{1}\PY{p}{]}\PY{p}{[}\PY{n}{y}\PY{o}{\PYZhy{}}\PY{n}{pos\PYZus{}y}\PY{o}{+}\PY{l+m+mi}{1}\PY{p}{]} \PY{o}{=} \PY{p}{[}\PY{p}{]}
    \PY{k}{return} \PY{n}{f\PYZus{}mat}
\end{Verbatim}
\end{tcolorbox}

    \hypertarget{implementacion-de-nodo}{%
\subsection{Implementacion de Nodo}\label{implementacion-de-nodo}}

La mejor forma de controlar las casillas o nodos, es creando un objeto
donde se pueda almacenar cual es el nodo padre y los valores de \(g\),
\(h\) y \(f\).

    \begin{tcolorbox}[breakable, size=fbox, boxrule=1pt, pad at break*=1mm,colback=cellbackground, colframe=cellborder]
\prompt{In}{incolor}{ }{\boxspacing}
\begin{Verbatim}[commandchars=\\\{\}]
\PY{k}{class} \PY{n+nc}{Node}\PY{p}{(}\PY{p}{)}\PY{p}{:}
    \PY{k}{def} \PY{n+nf+fm}{\PYZus{}\PYZus{}init\PYZus{}\PYZus{}}\PY{p}{(}\PY{n+nb+bp}{self}\PY{p}{,} \PY{n}{parent}\PY{o}{=}\PY{k+kc}{None}\PY{p}{,} \PY{n}{position}\PY{o}{=}\PY{k+kc}{None}\PY{p}{,} \PY{n}{g}\PY{o}{=}\PY{l+m+mi}{0}\PY{p}{,} \PY{n}{h}\PY{o}{=}\PY{l+m+mi}{0}\PY{p}{,} \PY{n}{f}\PY{o}{=}\PY{l+m+mi}{0}\PY{p}{)}\PY{p}{:}
        \PY{n+nb+bp}{self}\PY{o}{.}\PY{n}{parent} \PY{o}{=} \PY{n}{parent}
        \PY{n+nb+bp}{self}\PY{o}{.}\PY{n}{position} \PY{o}{=} \PY{n}{position} \PY{c+c1}{\PYZsh{}tupla (fila,columna) }
        
        \PY{n+nb+bp}{self}\PY{o}{.}\PY{n}{g} \PY{o}{=} \PY{n}{g}
        \PY{n+nb+bp}{self}\PY{o}{.}\PY{n}{h} \PY{o}{=} \PY{n}{h}
        \PY{n+nb+bp}{self}\PY{o}{.}\PY{n}{f} \PY{o}{=} \PY{n}{f}
        
    \PY{k}{def} \PY{n+nf+fm}{\PYZus{}\PYZus{}eq\PYZus{}\PYZus{}}\PY{p}{(}\PY{n+nb+bp}{self}\PY{p}{,} \PY{n}{other}\PY{p}{)}\PY{p}{:}
        \PY{k}{return} \PY{n+nb+bp}{self}\PY{p}{,}
\end{Verbatim}
\end{tcolorbox}

    \hypertarget{implementaciuxf3n-inventario}{%
\subsection{Implementación
Inventario}\label{implementaciuxf3n-inventario}}

En esta clase almacenamos la información basica de los inventarios, el
nombre, la posición inicial y destino.

    \begin{tcolorbox}[breakable, size=fbox, boxrule=1pt, pad at break*=1mm,colback=cellbackground, colframe=cellborder]
\prompt{In}{incolor}{ }{\boxspacing}
\begin{Verbatim}[commandchars=\\\{\}]
\PY{k}{class} \PY{n+nc}{Inventario}\PY{p}{(}\PY{p}{)}\PY{p}{:}
    \PY{k}{def} \PY{n+nf+fm}{\PYZus{}\PYZus{}init\PYZus{}\PYZus{}}\PY{p}{(}\PY{n+nb+bp}{self}\PY{p}{,} \PY{n}{nombre}\PY{p}{,} \PY{n}{pos\PYZus{}init}\PY{p}{,} \PY{n}{pos\PYZus{}dest}\PY{p}{,} \PY{n}{val}\PY{p}{)}\PY{p}{:}
        \PY{n+nb+bp}{self}\PY{o}{.}\PY{n}{nombre} \PY{o}{=} \PY{n}{nombre}
        \PY{n+nb+bp}{self}\PY{o}{.}\PY{n}{pos\PYZus{}init} \PY{o}{=} \PY{n}{pos\PYZus{}init}
        \PY{n+nb+bp}{self}\PY{o}{.}\PY{n}{pos\PYZus{}dest} \PY{o}{=} \PY{n}{pos\PYZus{}dest}
        \PY{n+nb+bp}{self}\PY{o}{.}\PY{n}{val} \PY{o}{=} \PY{n}{val}
\end{Verbatim}
\end{tcolorbox}

    \hypertarget{algoritmo-a}{%
\subsection{Algoritmo A*}\label{algoritmo-a}}

A continuación, se explica de forma detallada la lógica del algoritmo.

En esta parte del código obtenemos el primer nodo de la lista abierta,
luego en un ciclo buscamos en la lista abierta el nodo con menor valor
\(f\), y se define este nodo como el nodo actual. Y por último lo
eliminamos de la lista abierta y lo movemos a la lista cerrada.

\begin{verbatim}
# Obteniendo el nodo actual
current_node = open_list[0]
current_index = 0

#Encontrar el valor de f menor en la lista abierta
for index, item in enumerate(open_list):
    if item.f < current_node.f:
        current_node = item
        current_index = index

# Sacamos el nodo con menor f de la lista abierta 
# y lo agregamos a la cerrada
open_list.pop(current_index)
closed_list.append(current_node)
\end{verbatim}

Se valida si la posición actual es la posición final, si esa condición
se cumple tomamos el nodo actual y en un ciclo \(while\) obtenemos el
padre del ciclo hasta que retorne \(Null\), indicando que se llego al
nodo inicial. Cada uno de los padres se guarda en una lista y se
invierte en el \(return\) para obtener la lista de nodos ordenada, esta
lista indica el camino del nodo inicial a la posición final.

\begin{verbatim}
# Validar si es la posicion final
if current_node.position == end_node.position:
    path = []
    nodo = current_node
    # Recorremos cada nodo padre para obtener la ruta
    while nodo is not None:
        path.append(nodo)
        nodo = nodo.parent
        
    return path[::-1] # Retornamos la ruta
\end{verbatim}

Este es el paso final antes de repetir la iteración, utilizamos la
función \textbf{fgh\_values}, enviándole como parámetros el tablero
actual, la posición actual y el destino, la función nos retorna la
matriz con las celdas vecinas a las que es posible moverse con los
valores (f,g,h).

\begin{verbatim}
# Generar los hijos
values_matrix = fgh_values(board, target=end, position=current_node.position)
\end{verbatim}

Con los valores de las celdas cercanas vamos a seleccionar las celdas
validas que pueden ser \textbf{hijos}, para esta validación primero
buscamos que no existan en la lista abierta, si existe en la lista
abierta validamos si el valor de \(g\) actual es menor al que tiene el
nodo en la lista abierta, si mejora entonces actualizamos el nodo en la
lista abierta, cambiamos su nodo padre al nodo actual, y actualizamos
sus valores de \(f\) y \(g\), el valor \(h\) se mantiene igual ya que es
la distancia al objetivo.

\begin{verbatim}
children = []
for row in values_matrix:
    for cell in row:
        if len(cell) > 0:
            x = cell[3][0] #Columna
            y = cell[3][1] #Fila
            add_flag = True
            new_g = cell[1] + current_node.g
            new_f = new_g + cell[2]
            
            # Comprobando que no este en lista abierta
            for node in open_list:
                if node.position == (y,x):
                    add_flag = False
                    if node.g > new_g:
                        node.parent = current_node
                        node.g = new_g
                        node.f = new_f
                    
                    
            # Comprobando que no este en lista cerrada
            for node in closed_list:
                if node.position == (y,x):
                    add_flag = False
\end{verbatim}

Creamos los objetos nodos hijos y se agregan a la lista abierta.

\begin{verbatim}
            # Creamos un nuevo nodo
            new_node = Node(parent=current_node, position=(y,x), g=new_g, h=cell[2], f=new_f)
            if add_flag:
                children.append(new_node)

open_list.extend(children)
\end{verbatim}

    \hypertarget{implementacion-del-algoritmo}{%
\subsubsection{Implementacion del
Algoritmo}\label{implementacion-del-algoritmo}}

    \begin{tcolorbox}[breakable, size=fbox, boxrule=1pt, pad at break*=1mm,colback=cellbackground, colframe=cellborder]
\prompt{In}{incolor}{ }{\boxspacing}
\begin{Verbatim}[commandchars=\\\{\}]
\PY{k}{def} \PY{n+nf}{astar}\PY{p}{(}\PY{n}{board}\PY{p}{,} \PY{n}{start}\PY{p}{,} \PY{n}{end}\PY{p}{)}\PY{p}{:}
    \PY{c+c1}{\PYZsh{} Creando el nodo inicial y final}
    \PY{n}{start\PYZus{}node} \PY{o}{=} \PY{n}{Node}\PY{p}{(}\PY{n}{position}\PY{o}{=}\PY{n}{start}\PY{p}{)}
    \PY{n}{end\PYZus{}node} \PY{o}{=} \PY{n}{Node}\PY{p}{(}\PY{n}{position}\PY{o}{=}\PY{n}{end}\PY{p}{)}
    
    \PY{c+c1}{\PYZsh{} Inicializando lista cerrada y lista abierta}
    \PY{n}{open\PYZus{}list} \PY{o}{=} \PY{p}{[}\PY{p}{]}
    \PY{n}{closed\PYZus{}list} \PY{o}{=} \PY{p}{[}\PY{p}{]}
    
    \PY{c+c1}{\PYZsh{}Agregamos a la lista abierta el nodo inicial}
    \PY{n}{open\PYZus{}list}\PY{o}{.}\PY{n}{append}\PY{p}{(}\PY{n}{start\PYZus{}node}\PY{p}{)}
    
    \PY{k}{while} \PY{n+nb}{len}\PY{p}{(}\PY{n}{open\PYZus{}list}\PY{p}{)} \PY{o}{\PYZgt{}} \PY{l+m+mi}{0}\PY{p}{:}
        
        \PY{c+c1}{\PYZsh{} Obteniendo el nodo actual}
        \PY{n}{current\PYZus{}node} \PY{o}{=} \PY{n}{open\PYZus{}list}\PY{p}{[}\PY{l+m+mi}{0}\PY{p}{]}
        \PY{n}{current\PYZus{}index} \PY{o}{=} \PY{l+m+mi}{0}
        
        \PY{c+c1}{\PYZsh{}Encontrar el valor de f menor en la lista abierta}
        \PY{k}{for} \PY{n}{index}\PY{p}{,} \PY{n}{item} \PY{o+ow}{in} \PY{n+nb}{enumerate}\PY{p}{(}\PY{n}{open\PYZus{}list}\PY{p}{)}\PY{p}{:}
            \PY{k}{if} \PY{n}{item}\PY{o}{.}\PY{n}{f} \PY{o}{\PYZlt{}} \PY{n}{current\PYZus{}node}\PY{o}{.}\PY{n}{f}\PY{p}{:}
                \PY{n}{current\PYZus{}node} \PY{o}{=} \PY{n}{item}
                \PY{n}{current\PYZus{}index} \PY{o}{=} \PY{n}{index}
        
        \PY{c+c1}{\PYZsh{} Sacamos el nodo con menor f de la lista abierta }
        \PY{c+c1}{\PYZsh{} y lo agregamos a la cerrada}
        \PY{n}{open\PYZus{}list}\PY{o}{.}\PY{n}{pop}\PY{p}{(}\PY{n}{current\PYZus{}index}\PY{p}{)}
        \PY{n}{closed\PYZus{}list}\PY{o}{.}\PY{n}{append}\PY{p}{(}\PY{n}{current\PYZus{}node}\PY{p}{)}

        
        \PY{c+c1}{\PYZsh{} Validar si es la posicion final}
        \PY{k}{if} \PY{n}{current\PYZus{}node}\PY{o}{.}\PY{n}{position} \PY{o}{==} \PY{n}{end\PYZus{}node}\PY{o}{.}\PY{n}{position}\PY{p}{:}
            \PY{n}{path} \PY{o}{=} \PY{p}{[}\PY{p}{]}
            \PY{n}{nodo} \PY{o}{=} \PY{n}{current\PYZus{}node}
            \PY{c+c1}{\PYZsh{} Recorremos cada nodo padre para obtener la ruta}
            \PY{k}{while} \PY{n}{nodo} \PY{o+ow}{is} \PY{o+ow}{not} \PY{k+kc}{None}\PY{p}{:}
                \PY{n}{path}\PY{o}{.}\PY{n}{append}\PY{p}{(}\PY{n}{nodo}\PY{p}{)}
                \PY{n}{nodo} \PY{o}{=} \PY{n}{nodo}\PY{o}{.}\PY{n}{parent}
                
            \PY{k}{return} \PY{n}{path}\PY{p}{[}\PY{p}{:}\PY{p}{:}\PY{o}{\PYZhy{}}\PY{l+m+mi}{1}\PY{p}{]} \PY{c+c1}{\PYZsh{} Retornamos la ruta}
        
        
        \PY{c+c1}{\PYZsh{} Generar los hijos}
        \PY{n}{values\PYZus{}matrix} \PY{o}{=} \PY{n}{fgh\PYZus{}values}\PY{p}{(}\PY{n}{board}\PY{p}{,} \PY{n}{target}\PY{o}{=}\PY{n}{end}\PY{p}{,} \PY{n}{position}\PY{o}{=}\PY{n}{current\PYZus{}node}\PY{o}{.}\PY{n}{position}\PY{p}{)}

        \PY{n}{children} \PY{o}{=} \PY{p}{[}\PY{p}{]}
        \PY{k}{for} \PY{n}{row} \PY{o+ow}{in} \PY{n}{values\PYZus{}matrix}\PY{p}{:}
            \PY{k}{for} \PY{n}{cell} \PY{o+ow}{in} \PY{n}{row}\PY{p}{:}
                \PY{k}{if} \PY{n+nb}{len}\PY{p}{(}\PY{n}{cell}\PY{p}{)} \PY{o}{\PYZgt{}} \PY{l+m+mi}{0}\PY{p}{:}
                    \PY{n}{x} \PY{o}{=} \PY{n}{cell}\PY{p}{[}\PY{l+m+mi}{3}\PY{p}{]}\PY{p}{[}\PY{l+m+mi}{0}\PY{p}{]} \PY{c+c1}{\PYZsh{}Columna}
                    \PY{n}{y} \PY{o}{=} \PY{n}{cell}\PY{p}{[}\PY{l+m+mi}{3}\PY{p}{]}\PY{p}{[}\PY{l+m+mi}{1}\PY{p}{]} \PY{c+c1}{\PYZsh{}Fila}
                    \PY{n}{add\PYZus{}flag} \PY{o}{=} \PY{k+kc}{True}
                    \PY{n}{new\PYZus{}g} \PY{o}{=} \PY{n}{cell}\PY{p}{[}\PY{l+m+mi}{1}\PY{p}{]} \PY{o}{+} \PY{n}{current\PYZus{}node}\PY{o}{.}\PY{n}{g}
                    \PY{n}{new\PYZus{}f} \PY{o}{=} \PY{n}{new\PYZus{}g} \PY{o}{+} \PY{n}{cell}\PY{p}{[}\PY{l+m+mi}{2}\PY{p}{]}
                    
                    \PY{c+c1}{\PYZsh{} Comprobando que no este en lista abierta}
                    \PY{k}{for} \PY{n}{node} \PY{o+ow}{in} \PY{n}{open\PYZus{}list}\PY{p}{:}
                        \PY{k}{if} \PY{n}{node}\PY{o}{.}\PY{n}{position} \PY{o}{==} \PY{p}{(}\PY{n}{y}\PY{p}{,}\PY{n}{x}\PY{p}{)}\PY{p}{:}
                            \PY{n}{add\PYZus{}flag} \PY{o}{=} \PY{k+kc}{False}
                            \PY{c+c1}{\PYZsh{} Si el valor de g actual es menor cambiamos el }
                            \PY{c+c1}{\PYZsh{} nodo padre y sus valores de g,f}
                            \PY{k}{if} \PY{n}{node}\PY{o}{.}\PY{n}{g} \PY{o}{\PYZgt{}} \PY{n}{new\PYZus{}g}\PY{p}{:}
                                \PY{n}{node}\PY{o}{.}\PY{n}{parent} \PY{o}{=} \PY{n}{current\PYZus{}node}
                                \PY{n}{node}\PY{o}{.}\PY{n}{g} \PY{o}{=} \PY{n}{new\PYZus{}g}
                                \PY{n}{node}\PY{o}{.}\PY{n}{f} \PY{o}{=} \PY{n}{new\PYZus{}f}
                            
                            
                    \PY{c+c1}{\PYZsh{} Comprobando que no este en lista cerrada}
                    \PY{k}{for} \PY{n}{node} \PY{o+ow}{in} \PY{n}{closed\PYZus{}list}\PY{p}{:}
                        \PY{k}{if} \PY{n}{node}\PY{o}{.}\PY{n}{position} \PY{o}{==} \PY{p}{(}\PY{n}{y}\PY{p}{,}\PY{n}{x}\PY{p}{)}\PY{p}{:}
                            \PY{n}{add\PYZus{}flag} \PY{o}{=} \PY{k+kc}{False}
                            
                    \PY{c+c1}{\PYZsh{} Creamos un nuevo nodo}
                    \PY{n}{new\PYZus{}node} \PY{o}{=} \PY{n}{Node}\PY{p}{(}\PY{n}{parent}\PY{o}{=}\PY{n}{current\PYZus{}node}\PY{p}{,} \PY{n}{position}\PY{o}{=}\PY{p}{(}\PY{n}{y}\PY{p}{,}\PY{n}{x}\PY{p}{)}\PY{p}{,} \PY{n}{g}\PY{o}{=}\PY{n}{new\PYZus{}g}\PY{p}{,} \PY{n}{h}\PY{o}{=}\PY{n}{cell}\PY{p}{[}\PY{l+m+mi}{2}\PY{p}{]}\PY{p}{,} \PY{n}{f}\PY{o}{=}\PY{n}{new\PYZus{}f}\PY{p}{)}
                    \PY{k}{if} \PY{n}{add\PYZus{}flag}\PY{p}{:}
                        \PY{n}{children}\PY{o}{.}\PY{n}{append}\PY{p}{(}\PY{n}{new\PYZus{}node}\PY{p}{)}
        
        \PY{c+c1}{\PYZsh{} Agregamos a la lista abierta los nuevos nodos hijos}
        \PY{n}{open\PYZus{}list}\PY{o}{.}\PY{n}{extend}\PY{p}{(}\PY{n}{children}\PY{p}{)}      
                    
\end{Verbatim}
\end{tcolorbox}

    \hypertarget{funciones-para-graficar}{%
\paragraph{Funciones para graficar}\label{funciones-para-graficar}}

    \begin{tcolorbox}[breakable, size=fbox, boxrule=1pt, pad at break*=1mm,colback=cellbackground, colframe=cellborder]
\prompt{In}{incolor}{ }{\boxspacing}
\begin{Verbatim}[commandchars=\\\{\}]
\PY{n}{fig} \PY{o}{=} \PY{n}{plt}\PY{o}{.}\PY{n}{figure}\PY{p}{(}\PY{p}{)}
\PY{n}{ims} \PY{o}{=} \PY{p}{[}\PY{p}{]}

\PY{k}{def} \PY{n+nf}{mapping}\PY{p}{(}\PY{n}{option}\PY{o}{=}\PY{l+s+s1}{\PYZsq{}}\PY{l+s+s1}{\PYZsq{}}\PY{p}{,}\PY{n}{matrix}\PY{o}{=}\PY{p}{[}\PY{p}{]}\PY{p}{,} \PY{n}{nombre}\PY{o}{=}\PY{l+s+s1}{\PYZsq{}}\PY{l+s+s1}{\PYZsq{}}\PY{p}{)}\PY{p}{:}
    \PY{k}{global} \PY{n}{ims}

    \PY{k}{if}\PY{p}{(}\PY{n}{option}\PY{o}{==}\PY{l+s+s1}{\PYZsq{}}\PY{l+s+s1}{\PYZsq{}}\PY{p}{)}\PY{p}{:}
        \PY{k}{pass}
    \PY{k}{elif}\PY{p}{(}\PY{n}{option}\PY{o}{==}\PY{l+s+s1}{\PYZsq{}}\PY{l+s+s1}{add}\PY{l+s+s1}{\PYZsq{}}\PY{p}{)}\PY{p}{:}
        \PY{n}{im} \PY{o}{=} \PY{n}{plt}\PY{o}{.}\PY{n}{imshow}\PY{p}{(}\PY{n}{matrix}\PY{p}{,} \PY{n}{cmap} \PY{o}{=} \PY{n}{plt}\PY{o}{.}\PY{n}{cm}\PY{o}{.}\PY{n}{gray}\PY{p}{)}
        \PY{n}{ims}\PY{o}{.}\PY{n}{append}\PY{p}{(}\PY{p}{[}\PY{n}{im}\PY{p}{]}\PY{p}{)}
    \PY{k}{elif}\PY{p}{(}\PY{n}{option}\PY{o}{==}\PY{l+s+s1}{\PYZsq{}}\PY{l+s+s1}{plot}\PY{l+s+s1}{\PYZsq{}}\PY{p}{)}\PY{p}{:}
        \PY{n}{ani} \PY{o}{=} \PY{n}{animation}\PY{o}{.}\PY{n}{ArtistAnimation}\PY{p}{(}\PY{n}{fig}\PY{p}{,} \PY{n}{ims}\PY{p}{,} \PY{n}{interval}\PY{o}{=}\PY{l+m+mi}{500}\PY{p}{,} \PY{n}{blit}\PY{o}{=}\PY{k+kc}{True}\PY{p}{,}\PY{n}{repeat\PYZus{}delay}\PY{o}{=}\PY{l+m+mi}{100}\PY{p}{)}
        \PY{n}{ani}\PY{o}{.}\PY{n}{save}\PY{p}{(}\PY{n}{nombre}\PY{p}{,} \PY{n}{writer}\PY{o}{=}\PY{l+s+s1}{\PYZsq{}}\PY{l+s+s1}{imagemagick}\PY{l+s+s1}{\PYZsq{}}\PY{p}{,} \PY{n}{fps}\PY{o}{=}\PY{l+m+mi}{4}\PY{p}{)}

        
\PY{k}{def} \PY{n+nf}{update\PYZus{}matrix}\PY{p}{(}\PY{n}{current\PYZus{}matrix}\PY{p}{,} \PY{n}{curr\PYZus{}pos}\PY{p}{,}\PY{n}{new\PYZus{}pos}\PY{p}{,} \PY{n}{obj\PYZus{}val}\PY{p}{,} \PY{n}{begin\PYZus{}load\PYZus{}route}\PY{o}{=}\PY{k+kc}{False}\PY{p}{)}\PY{p}{:}
    \PY{c+c1}{\PYZsh{}Actualizar Tablero}
    \PY{k}{if} \PY{p}{(}\PY{n}{begin\PYZus{}load\PYZus{}route} \PY{o+ow}{is} \PY{k+kc}{False}\PY{p}{)}\PY{p}{:}
        \PY{n}{current\PYZus{}matrix}\PY{p}{[}\PY{n}{curr\PYZus{}pos}\PY{p}{]} \PY{o}{=} \PY{n}{object\PYZus{}num\PYZus{}mapping}\PY{p}{[}\PY{l+s+s1}{\PYZsq{}}\PY{l+s+s1}{espacio}\PY{l+s+s1}{\PYZsq{}}\PY{p}{]}
    \PY{n}{current\PYZus{}matrix}\PY{p}{[}\PY{n}{new\PYZus{}pos}\PY{p}{]} \PY{o}{=} \PY{n}{obj\PYZus{}val}
    \PY{k}{return} \PY{n}{current\PYZus{}matrix}
\end{Verbatim}
\end{tcolorbox}

    
    \begin{Verbatim}[commandchars=\\\{\}]
<Figure size 432x288 with 0 Axes>
    \end{Verbatim}

    
    \hypertarget{ejecutando-algoritmo-a}{%
\subsection{Ejecutando Algoritmo A*}\label{ejecutando-algoritmo-a}}

    \hypertarget{inicializando-el-tablero}{%
\subsubsection{Inicializando el
tablero}\label{inicializando-el-tablero}}

    \begin{tcolorbox}[breakable, size=fbox, boxrule=1pt, pad at break*=1mm,colback=cellbackground, colframe=cellborder]
\prompt{In}{incolor}{ }{\boxspacing}
\begin{Verbatim}[commandchars=\\\{\}]
\PY{c+c1}{\PYZsh{}Matriz de estado inicial con }
\PY{n}{initial\PYZus{}state} \PY{o}{=} \PY{n}{np}\PY{o}{.}\PY{n}{matrix}\PY{p}{(}\PY{p}{[}\PY{p}{[}\PY{l+m+mi}{6}\PY{p}{,} \PY{l+m+mi}{1}\PY{p}{,} \PY{l+m+mi}{0}\PY{p}{,} \PY{l+m+mi}{5}\PY{p}{]}\PY{p}{,}
                           \PY{p}{[}\PY{l+m+mi}{0}\PY{p}{,} \PY{l+m+mi}{1}\PY{p}{,} \PY{l+m+mi}{0}\PY{p}{,} \PY{l+m+mi}{0}\PY{p}{]}\PY{p}{,}
                           \PY{p}{[}\PY{l+m+mi}{4}\PY{p}{,} \PY{l+m+mi}{0}\PY{p}{,} \PY{l+m+mi}{2}\PY{p}{,} \PY{l+m+mi}{0}\PY{p}{]}\PY{p}{,}
                           \PY{p}{[}\PY{l+m+mi}{0}\PY{p}{,} \PY{l+m+mi}{0}\PY{p}{,} \PY{l+m+mi}{0}\PY{p}{,} \PY{l+m+mi}{0}\PY{p}{]}\PY{p}{]}\PY{p}{)}
\end{Verbatim}
\end{tcolorbox}

    \hypertarget{generacion-de-inventarios}{%
\subsubsection{Generacion de
Inventarios}\label{generacion-de-inventarios}}

Generamos 3 objetos de tipo inventarios con le enviamos como parametro
la posicion inicial y la posicion final.

    \begin{tcolorbox}[breakable, size=fbox, boxrule=1pt, pad at break*=1mm,colback=cellbackground, colframe=cellborder]
\prompt{In}{incolor}{ }{\boxspacing}
\begin{Verbatim}[commandchars=\\\{\}]
\PY{n}{inventarios} \PY{o}{=} \PY{p}{[}\PY{p}{]}

\PY{c+c1}{\PYZsh{} Inventario M3}
\PY{n}{new\PYZus{}invent} \PY{o}{=} \PY{n}{Inventario}\PY{p}{(}\PY{l+s+s1}{\PYZsq{}}\PY{l+s+s1}{M3}\PY{l+s+s1}{\PYZsq{}}\PY{p}{,} \PY{n}{pos\PYZus{}init}\PY{o}{=}\PY{p}{(}\PY{l+m+mi}{0}\PY{p}{,}\PY{l+m+mi}{3}\PY{p}{)}\PY{p}{,} \PY{n}{pos\PYZus{}dest}\PY{o}{=}\PY{p}{(}\PY{l+m+mi}{3}\PY{p}{,}\PY{l+m+mi}{1}\PY{p}{)}\PY{p}{,} \PY{n}{val}\PY{o}{=}\PY{l+m+mi}{5}\PY{p}{)}
\PY{n}{inventarios}\PY{o}{.}\PY{n}{append}\PY{p}{(}\PY{n}{new\PYZus{}invent}\PY{p}{)}

\PY{c+c1}{\PYZsh{} Inventario M2}
\PY{n}{new\PYZus{}invent} \PY{o}{=} \PY{n}{Inventario}\PY{p}{(}\PY{l+s+s1}{\PYZsq{}}\PY{l+s+s1}{M2}\PY{l+s+s1}{\PYZsq{}}\PY{p}{,} \PY{n}{pos\PYZus{}init}\PY{o}{=}\PY{p}{(}\PY{l+m+mi}{2}\PY{p}{,}\PY{l+m+mi}{0}\PY{p}{)}\PY{p}{,} \PY{n}{pos\PYZus{}dest}\PY{o}{=}\PY{p}{(}\PY{l+m+mi}{3}\PY{p}{,}\PY{l+m+mi}{2}\PY{p}{)}\PY{p}{,} \PY{n}{val}\PY{o}{=}\PY{l+m+mi}{4}\PY{p}{)}
\PY{n}{inventarios}\PY{o}{.}\PY{n}{append}\PY{p}{(}\PY{n}{new\PYZus{}invent}\PY{p}{)}

\PY{c+c1}{\PYZsh{} Inventario M1}
\PY{n}{new\PYZus{}invent} \PY{o}{=} \PY{n}{Inventario}\PY{p}{(}\PY{l+s+s1}{\PYZsq{}}\PY{l+s+s1}{M1}\PY{l+s+s1}{\PYZsq{}}\PY{p}{,} \PY{n}{pos\PYZus{}init}\PY{o}{=}\PY{p}{(}\PY{l+m+mi}{0}\PY{p}{,}\PY{l+m+mi}{0}\PY{p}{)}\PY{p}{,} \PY{n}{pos\PYZus{}dest}\PY{o}{=}\PY{p}{(}\PY{l+m+mi}{3}\PY{p}{,}\PY{l+m+mi}{3}\PY{p}{)}\PY{p}{,} \PY{n}{val}\PY{o}{=}\PY{l+m+mi}{6}\PY{p}{)}
\PY{n}{inventarios}\PY{o}{.}\PY{n}{append}\PY{p}{(}\PY{n}{new\PYZus{}invent}\PY{p}{)}
\end{Verbatim}
\end{tcolorbox}

    \hypertarget{ejecuciuxf3n}{%
\subsubsection{Ejecución}\label{ejecuciuxf3n}}

Se realizo una búsqueda por sub-objetivos, tenemos un arreglo de
inventarios, utilizamos un ciclo \(for\) para recorrer el arreglo de
inventarios. Con el inventario actual se realizan los siguientes pasos:
1. Ejecutamos el algoritmo A* para que el robot pueda llegar se su
posición actual al inventario, se imprime la ruta. 2. Ejecutamos el
algoritmo A* de nuevo para que el robot lleve el inventario a su
posición destino y se imprime la ruta.

    \begin{tcolorbox}[breakable, size=fbox, boxrule=1pt, pad at break*=1mm,colback=cellbackground, colframe=cellborder]
\prompt{In}{incolor}{ }{\boxspacing}
\begin{Verbatim}[commandchars=\\\{\}]
\PY{n}{pos\PYZus{}robot} \PY{o}{=} \PY{p}{(}\PY{l+m+mi}{2}\PY{p}{,}\PY{l+m+mi}{2}\PY{p}{)}
\PY{n}{board} \PY{o}{=} \PY{n}{initial\PYZus{}state}

\PY{n}{begin\PYZus{}route\PYZus{}after\PYZus{}load} \PY{o}{=} \PY{k+kc}{False}
\PY{k}{for} \PY{n}{inventario} \PY{o+ow}{in} \PY{n}{inventarios} \PY{p}{:}
    \PY{n+nb}{print}\PY{p}{(}\PY{l+s+sa}{f}\PY{l+s+s2}{\PYZdq{}}\PY{l+s+s2}{==================== Inventario }\PY{l+s+si}{\PYZob{}}\PY{n}{inventario}\PY{o}{.}\PY{n}{nombre}\PY{l+s+si}{\PYZcb{}}\PY{l+s+s2}{ ===========================}\PY{l+s+s2}{\PYZdq{}}\PY{p}{)}
    \PY{n+nb}{print}\PY{p}{(}\PY{l+s+sa}{f}\PY{l+s+s2}{\PYZdq{}}\PY{l+s+s2}{ Robot Posicion actual }\PY{l+s+si}{\PYZob{}}\PY{n}{pos\PYZus{}robot}\PY{l+s+si}{\PYZcb{}}\PY{l+s+s2}{\PYZdq{}}\PY{p}{)}

    \PY{c+c1}{\PYZsh{}\PYZsh{} Calcular ruta del robot \PYZhy{}\PYZgt{} inventario}
    \PY{c+c1}{\PYZsh{}\PYZsh{} Algoritmo A*}
    \PY{n}{ruta1} \PY{o}{=} \PY{n}{astar}\PY{p}{(}\PY{n}{board}\PY{p}{,} \PY{n}{pos\PYZus{}robot}\PY{p}{,} \PY{n}{inventario}\PY{o}{.}\PY{n}{pos\PYZus{}init}\PY{p}{)}
    \PY{n+nb}{print}\PY{p}{(}\PY{l+s+sa}{f}\PY{l+s+s2}{\PYZdq{}}\PY{l+s+s2}{=\PYZgt{} Ruta del robot al inventario }\PY{l+s+si}{\PYZob{}}\PY{n}{inventario}\PY{o}{.}\PY{n}{nombre}\PY{l+s+si}{\PYZcb{}}\PY{l+s+s2}{\PYZdq{}}\PY{p}{)}
    \PY{n}{last\PYZus{}pos} \PY{o}{=} \PY{n}{pos\PYZus{}robot}
    
    \PY{c+c1}{\PYZsh{}\PYZsh{} Imprimiendo la ruta}
    \PY{k}{for} \PY{n}{i}\PY{p}{,} \PY{n}{nodo} \PY{o+ow}{in} \PY{n+nb}{enumerate}\PY{p}{(}\PY{n}{ruta1}\PY{p}{)}\PY{p}{:}
        \PY{k}{if} \PY{p}{(}\PY{n}{nodo}\PY{o}{.}\PY{n}{position} \PY{o}{!=} \PY{n}{last\PYZus{}pos}\PY{p}{)}\PY{p}{:}
            \PY{n}{board} \PY{o}{=} \PY{n}{update\PYZus{}matrix}\PY{p}{(}\PY{n}{board}\PY{p}{,} \PY{n}{last\PYZus{}pos}\PY{p}{,} \PY{n}{nodo}\PY{o}{.}\PY{n}{position}\PY{p}{,} \PY{n}{object\PYZus{}num\PYZus{}mapping}\PY{p}{[}\PY{l+s+s1}{\PYZsq{}}\PY{l+s+s1}{robot}\PY{l+s+s1}{\PYZsq{}}\PY{p}{]}\PY{p}{,} \PY{n}{begin\PYZus{}route\PYZus{}after\PYZus{}load}\PY{p}{)}
            \PY{n}{begin\PYZus{}route\PYZus{}after\PYZus{}load} \PY{o}{=} \PY{k+kc}{False}
        \PY{n}{last\PYZus{}pos} \PY{o}{=} \PY{n}{nodo}\PY{o}{.}\PY{n}{position}
        \PY{n+nb}{print}\PY{p}{(}\PY{l+s+sa}{f}\PY{l+s+s2}{\PYZdq{}}\PY{l+s+s2}{Mover R fila: }\PY{l+s+si}{\PYZob{}}\PY{n}{nodo}\PY{o}{.}\PY{n}{position}\PY{p}{[}\PY{l+m+mi}{0}\PY{p}{]}\PY{l+s+si}{\PYZcb{}}\PY{l+s+s2}{, columna: }\PY{l+s+si}{\PYZob{}}\PY{n}{nodo}\PY{o}{.}\PY{n}{position}\PY{p}{[}\PY{l+m+mi}{1}\PY{p}{]}\PY{l+s+si}{\PYZcb{}}\PY{l+s+s2}{\PYZdq{}}\PY{p}{)}

        
    \PY{c+c1}{\PYZsh{}\PYZsh{} Calcular ruta del estante \PYZhy{}\PYZgt{} pos. final estante}
    \PY{c+c1}{\PYZsh{}\PYZsh{} Algoritmo A*}
    \PY{n}{ruta2} \PY{o}{=} \PY{n}{astar}\PY{p}{(}\PY{n}{board}\PY{p}{,} \PY{n}{inventario}\PY{o}{.}\PY{n}{pos\PYZus{}init}\PY{p}{,} \PY{n}{inventario}\PY{o}{.}\PY{n}{pos\PYZus{}dest}\PY{p}{)}
    \PY{n+nb}{print}\PY{p}{(}\PY{l+s+sa}{f}\PY{l+s+s2}{\PYZdq{}}\PY{l+s+s2}{=\PYZgt{} Ruta del inventario }\PY{l+s+si}{\PYZob{}}\PY{n}{inventario}\PY{o}{.}\PY{n}{nombre}\PY{l+s+si}{\PYZcb{}}\PY{l+s+s2}{ a destino }\PY{l+s+si}{\PYZob{}}\PY{n}{inventario}\PY{o}{.}\PY{n}{pos\PYZus{}dest}\PY{l+s+si}{\PYZcb{}}\PY{l+s+s2}{\PYZdq{}}\PY{p}{)}
    \PY{n}{last\PYZus{}pos} \PY{o}{=} \PY{n}{inventario}\PY{o}{.}\PY{n}{pos\PYZus{}init}
    
    \PY{c+c1}{\PYZsh{}\PYZsh{} Imprimiendo la ruta}
    \PY{k}{for} \PY{n}{nodo} \PY{o+ow}{in} \PY{n}{ruta2}\PY{p}{:}
        \PY{k}{if} \PY{p}{(}\PY{n}{nodo}\PY{o}{.}\PY{n}{position} \PY{o}{!=} \PY{n}{last\PYZus{}pos}\PY{p}{)}\PY{p}{:}
            \PY{n}{board} \PY{o}{=} \PY{n}{update\PYZus{}matrix}\PY{p}{(}\PY{n}{board}\PY{p}{,} \PY{n}{last\PYZus{}pos}\PY{p}{,} \PY{n}{nodo}\PY{o}{.}\PY{n}{position}\PY{p}{,} \PY{n}{object\PYZus{}num\PYZus{}mapping}\PY{p}{[}\PY{l+s+s1}{\PYZsq{}}\PY{l+s+s1}{robot}\PY{l+s+s1}{\PYZsq{}}\PY{p}{]}\PY{p}{)}
        \PY{n}{last\PYZus{}pos} \PY{o}{=} \PY{n}{nodo}\PY{o}{.}\PY{n}{position}
        \PY{n+nb}{print}\PY{p}{(}\PY{l+s+sa}{f}\PY{l+s+s2}{\PYZdq{}}\PY{l+s+s2}{Cargar R[}\PY{l+s+si}{\PYZob{}}\PY{n}{inventario}\PY{o}{.}\PY{n}{nombre}\PY{l+s+si}{\PYZcb{}}\PY{l+s+s2}{] fila: }\PY{l+s+si}{\PYZob{}}\PY{n}{nodo}\PY{o}{.}\PY{n}{position}\PY{p}{[}\PY{l+m+mi}{0}\PY{p}{]}\PY{l+s+si}{\PYZcb{}}\PY{l+s+s2}{, columna: }\PY{l+s+si}{\PYZob{}}\PY{n}{nodo}\PY{o}{.}\PY{n}{position}\PY{p}{[}\PY{l+m+mi}{1}\PY{p}{]}\PY{l+s+si}{\PYZcb{}}\PY{l+s+s2}{\PYZdq{}}\PY{p}{)}
  

    \PY{n+nb}{print}\PY{p}{(}\PY{n}{board}\PY{p}{)}
    \PY{n}{begin\PYZus{}route\PYZus{}after\PYZus{}load} \PY{o}{=} \PY{k+kc}{True}
    \PY{n}{pos\PYZus{}robot} \PY{o}{=} \PY{n}{inventario}\PY{o}{.}\PY{n}{pos\PYZus{}dest}
\end{Verbatim}
\end{tcolorbox}

    \begin{Verbatim}[commandchars=\\\{\}]
==================== Inventario M3 ===========================
 Robot Posicion actual (2, 2)
=> Ruta del robot al inventario M3
Mover R fila: 2, columna: 2
Mover R fila: 1, columna: 3
Mover R fila: 0, columna: 3
=> Ruta del inventario M3 a destino (3, 1)
Cargar R[M3] fila: 0, columna: 3
Cargar R[M3] fila: 1, columna: 2
Cargar R[M3] fila: 2, columna: 1
Cargar R[M3] fila: 3, columna: 1
[[6 1 0 0]
 [0 1 0 0]
 [4 0 0 0]
 [0 2 0 0]]
==================== Inventario M2 ===========================
 Robot Posicion actual (3, 1)
=> Ruta del robot al inventario M2
Mover R fila: 3, columna: 1
Mover R fila: 2, columna: 0
=> Ruta del inventario M2 a destino (3, 2)
Cargar R[M2] fila: 2, columna: 0
Cargar R[M2] fila: 2, columna: 1
Cargar R[M2] fila: 3, columna: 2
[[6 1 0 0]
 [0 1 0 0]
 [0 0 0 0]
 [0 2 2 0]]
==================== Inventario M1 ===========================
 Robot Posicion actual (3, 2)
=> Ruta del robot al inventario M1
Mover R fila: 3, columna: 2
Mover R fila: 2, columna: 1
Mover R fila: 1, columna: 0
Mover R fila: 0, columna: 0
=> Ruta del inventario M1 a destino (3, 3)
Cargar R[M1] fila: 0, columna: 0
Cargar R[M1] fila: 1, columna: 0
Cargar R[M1] fila: 2, columna: 1
Cargar R[M1] fila: 2, columna: 2
Cargar R[M1] fila: 3, columna: 3
[[0 1 0 0]
 [0 1 0 0]
 [0 0 0 0]
 [0 2 2 2]]
    \end{Verbatim}

    \hypertarget{dificultades-encontradas}{%
\section{Dificultades Encontradas}\label{dificultades-encontradas}}

\begin{enumerate}
\def\labelenumi{\arabic{enumi}.}
\tightlist
\item
  El algoritmo es sencillo de comprender, sin embargo, requiere trabajo
  trasladarlo a código y que se encuentre bien estructurado.
\item
  Una dificultad visible en este código, es que se puede llegar a
  generar una cantidad muy alta de tiempo de ejecución por funciones
  asintóticas, dado que tiene algunos ciclos anidados en donde el tiempo
  de ejecución se puede tornar exponencial en casos de matrices muy
  grandes.
\item
  El algoritmo A*recorre una cuadrícula posición por posición hasta
  llegar a su punto final. Si se crea una matriz muy pequeña, el
  algoritmo puede generar un trazo con una incerteza muy alta en
  comparación con un terreno cuadriculado. Por otro lado, si se crea una
  matriz muy grande, se disminuye el error en una posición, pero se
  aumenta el procesamiento.
\end{enumerate}

    \hypertarget{referencias}{%
\section{Referencias}\label{referencias}}

\begin{enumerate}
\def\labelenumi{\arabic{enumi}.}
\tightlist
\item
  Roy, B. (2020, February 23). A-Star (A*) Search Algorithm - Towards
  Data Science. Medium.
  https://towardsdatascience.com/a-star-a-search-algorithm-eb495fb156bb
\item
  gammafp. (2017, July 27). Pathfinding A* (A estrella) - Tutorial
  completo en español {[}Video{]}. YouTube.
  https://www.youtube.com/watch?v=X-5JMScsZ14\&t=1363s
\end{enumerate}

    \begin{tcolorbox}[breakable, size=fbox, boxrule=1pt, pad at break*=1mm,colback=cellbackground, colframe=cellborder]
\prompt{In}{incolor}{ }{\boxspacing}
\begin{Verbatim}[commandchars=\\\{\}]

\end{Verbatim}
\end{tcolorbox}


    % Add a bibliography block to the postdoc
    
    
    
\end{document}
